\documentclass[10pt, oneside]{article}
\usepackage{amsmath}
\usepackage{amssymb}
\usepackage[utf8]{inputenc}
\usepackage[english]{babel}
\usepackage{titling}
\usepackage[nottoc, notlof]{tocbibind}
\usepackage[pdftex]{graphicx}
\usepackage[kerning,spacing]{microtype}
\usepackage{verbatim}
\usepackage{tikz}
\usetikzlibrary{arrows}

\usepackage[bookmarksnumbered, unicode, pdftex]{hyperref}

\author{Mikhail Glushenkov, \texttt{<c05mgv@cs.umu.se>}\\
        Bertil Nilsson, \texttt{<id09bnn@cs.umu.se>}}

\title{Assignment 2 -- GCom Middleware:\\Analysis and Design}

\newcommand{\unit}[1]{\ensuremath{\, \mathrm{#1}}}

\begin{document}
\pagestyle{plain}
\pdfbookmark[1]{Front page}{beg}

\begin{titlingpage}
  \begin{minipage}[t]{0.45\textwidth}
  \begin{flushleft}
  \texttt{5DV020 - Distributed Systems, Autumn 12}
  \end{flushleft}
  \end{minipage}
  \begin{minipage}[t]{0.4\textwidth}
  \begin{flushright}
  \texttt{Umeå University}
  \end{flushright}
  \end{minipage}
  \vskip 60pt
  \begin{center}
  \LARGE\thetitle
  \par\end{center}\vskip 0.5em
  \begin{center}
  \large\theauthor
  \par\end{center}
  \begin{center}
  Date: \today
  \par\end{center}
  \vfill
  \begin{center}
    \textbf{Instructors} \linebreak \linebreak
    Francisco Hernandez-Rodriguez\\
    Ewnetu Bayuh Lakew
  \end{center}
\end{titlingpage}

% TOC
%\thispagestyle{empty}
%\pagebreak
%\setcounter{page}{0}
%\pdfbookmark[1]{Table of contents}{tab}
%\tableofcontents
\pagebreak

% % i Sverige har vi normalt inget indrag vid nytt stycke
\setlength{\parindent}{0pt}
% men däremot lite mellanrum
\setlength{\parskip}{10pt}

\setcounter{section}{-1}

\section{Introduction}

The purpose of this assignment is to design and implement GCom, a middleware for
group communication. \ldots

\section{Requirements Specification}

The system will implement the L2 specification (dynamic groups).

Communication module:
\begin{itemize}
\item Basic non-reliable multicast
\item Basic reliable multicast
\end{itemize}

Message ordering module:
\begin{itemize}
\item Non-ordered
\item FIFO
\item Causal
\item Total
\item Causal-Total
\end{itemize}

Group management module:
\begin{itemize}
\item Create and remove groups
\item Add and remove group members
\item Monitoring of live processes
\item Notification about changes in group membership (incl. crashed nodes)
\item Group names resolution: a mapping group name $\rightarrow$ group members.
\end{itemize}

Test program:
\begin{itemize}
\item A simple distributed GUI chat program. Each chat client instance is a node
  of the distributed system.
\end{itemize}

Debug application:
\begin{itemize}
\item A malicious client for the aforementioned chat system.
\item Simulates packet loss (send a message to a part of the group).
\item Simulates packet rearrangement (send several messages in random order to
  different nodes).
\item Allows to choose message ordering.
\item Allows to view the internal state.
\item Allows to view a message's path.
\item Provides a measure of system performance.
\end{itemize}

\section{Design}

We decided to implement our system in Scala and use Java RMI (?) for
communication between processes. The choice of Scala was motivated by the desire
to use a language that is more modern and convenient than Java, but still allows
to run on JVM. The choice (?) of Java RMI over other alternatives was motivated
by \ldots (other alternatives: Twitter Finagle, ZeroMQ, \ldots)

Design of the communication module: simple, just use the reliable multicast
algorithm from the textbook.

Design of the message ordering module:
\begin{itemize}
\item FIFO: timestamps
\item Causal: vector clocks
\item Total: sequencer node chosen with an election alfgorithm
\item Causal-Total: combination of causal and total
\end{itemize}

Design of the group management module:
\begin{itemize}
\item A node is identified by a (host, port) pair (or (host, name) for RMI).
\item A group is identified by a group name.
\item Group names are globally unique.
\item Each node knows all group names, but not members
\item Each node knows a distinguished group node for each group
\item Distinguished node resolves group name to the list of group members
\item If a distinguished node dies, a new distinguished node is elected and
  everyone is notified
\item Join a group: get a list of group members, broadcast a notification to
  group members.
\item Leave a group: broadcast a notification.
\item If a node notices that one of its peers has died: broadcast a notification
  to the members of all groups it belonged to.
\end{itemize}


\section{Time Plan}

\begin{tabular}{|l|p{10cm}|}
\hline
Week 49 & Implement the communication module. Start working on the GUI test programs. \\
\hline
Week 50 & Implement the message ordering module. \\
\hline
Week 51 & Implement the group management module. \\
\hline
Week 52 & Testing and bugfixing. Start writing the final report. \\
\hline
Week 1  & More testing and bugfixing. Time for any unforeseen delays. \\
\hline
Week 2  & Practical demonstration. Turn in the final report. \\
\hline
\end{tabular}

\begin{thebibliography}{9}

\bibitem{Textbook} \emph{Distributed Systems}\\
\newblock George Coulouris, Jean Dollimore, Tim Kindberg and Gordon Blair\\
\newblock Addison-Wesley, 2011\\

\bibitem{Birman} \emph{Guide to Reliable Distributed Systems -- Building
    High-Assurance Applications and Cloud-Hosted Services}\\
\newblock Kenneth Birman\\
\newblock Springer, 2012\\

\end{thebibliography}


\end{document}
